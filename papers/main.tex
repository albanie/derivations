\documentclass[11pt,openany]{memoir}
\usepackage[usenames,dvipsnames]{xcolor}
\usepackage{amssymb}
\usepackage{amsmath}
\usepackage{amsthm}
\usepackage{url}
\usepackage{xspace}
\usepackage[margin=2.5cm]{geometry}
\usepackage{tikz}
\usetikzlibrary{positioning,calc,matrix,arrows}
%\usepackage{pgfplots}
\usepackage{listings}
\usepackage{color}
\usepackage{textcomp}
\usepackage{fancyvrb}
\usepackage{array}
\usepackage{dirtytalk}
\usepackage[colorlinks]{hyperref}
\usepackage{cleveref}
\usepackage{amsmath}

% NOTE: This template is based on the MatConvNet manual written by Andrea Vedaldi, Karel Lenc and Ankush Gupta.

\newtheorem{example}{Example}
\newtheorem{lemma}{Lemma}

\newcommand{\real}{\mathbb{R}}
\newcommand{\xx}{\mathbf{x}}
\newcommand{\yy}{\mathbf{y}}
\newcommand{\ww}{\mathbf{w}}
\newcommand{\pp}{\mathbf{p}}
\newcommand{\hh}{\mathbf{h}}
\newcommand{\ba}{\mathbf{a}}
\newcommand{\bb}{\mathbf{b}}
\newcommand{\bone}{\mathbf{1}}
\newcommand{\diag}{\operatorname{diag}}
\newcommand{\bone}{\mathbf{1}}
\newcommand{\FF}{\mathcal{F}}
\newcommand{\ththeta}{\pmb{\theta}}
\newcommand{\EE}{\mathbb{E}}
\DeclareMathOperator{\dd}{\text{d}}
\newcommand{\vv}{\operatorname{vec}}
\newcommand{\tr}{\operatorname{tr}}
%\newcommand{\by}{\mathbf{y}}
%\newcommand{\bc}{\mathbf{c}}
%\newcommand{\bz}{\mathbf{z}}
%\newcommand{\bff}{\mathbf{f}}
%\newcommand{\bg}{\mathbf{g}}
%\newcommand{\br}{\mathbf{r}}
%\newcommand{\bw}{\mathbf{w}}
%\newcommand{\bp}{\mathbf{p}}
%\newcommand{\bfs}{\mathbf{s}}
%\newcommand{\bfe}{\mathbf{e}}
\newcommand{\papersays}[1]{\textcolor{RubineRed}{#1}}

\setlength{\parindent}{0pt}

%\newcommand{\argmin}{\operatornamewithlimits{argmin}}
%\newcommand{\argmax}{\operatornamewithlimits{argmax}}
%\newcommand{\sign}{\operatornamewithlimits{sign}}
%
\tikzstyle{block} = [draw, rectangle, minimum height=3em, minimum width=3em]
\tikzstyle{data} = []
\tikzstyle{datac} = [draw, circle, minimum height=2.5em, minimum width=2.5em,inner sep=3pt,font=\footnotesize]
\tikzstyle{par} = [draw, circle, minimum height=2.5em, minimum width=2.5em,fill=black!20,inner sep=3pt,font=\footnotesize]
\tikzstyle{pinstyle} = [pin edge={to-,thin,black}]
\tikzstyle{to} = [->,>=stealth',shorten >=1pt,semithick]
\tikzstyle{from} = [<-,>=stealth',shorten >=1pt,semithick]
\tikzstyle{bp} = [draw=blue,text=blue]
\tikzstyle{bpl} = [draw=blue!40]
\tikzstyle{bpe} = [text=blue,draw=none]

\VerbatimFootnotes

\setsecnumdepth{subsection}
\settocdepth{subsection}

\definecolor{listinggray}{gray}{0.9}
\definecolor{lbcolor}{rgb}{0.8,0.8,0.8}
\lstset{
	%backgroundcolor=\color{lbcolor},
	tabsize=4,
	%rulecolor=,
	language=matlab,
	basicstyle=\small,
	%upquote=true,
	%columns=fullflexible,
	%showstringspaces=false,
	extendedchars=true,
	%breaklines=false,
	%prebreak = \raisebox{0ex}[0ex][0ex]{\ensuremath{\hookleftarrow}},
	%frame=single,
	%showtabs=false,
	%showspaces=false,
	%showstringspaces=false,
	%identifierstyle=\ttfamily,
	keywordstyle=\color[rgb]{0,0,1},
	commentstyle=\itshape\color[rgb]{0.133,0.545,0.133},
	stringstyle=\color[rgb]{0.627,0.126,0.941},
}
%\lstMakeShortInline[columns=fullflexible, breaklines = true,
% breakatwhitespace = true]!

\title{Derivations from research papers}
\author{
Samuel Albanie
}
\date{}

% REMOVE TO FIX COMPILATION
\hypersetup{
  pdfinfo={
    Title={Optimization for Applied Computer Vision},
    Author={Samuel Albanie},
        Subject={Notes on Mathematics},
    Keywords={computer vision, differentials, machine learning}
  }
}

\sloppy

% ------------------------------------------------------------------
\begin{document}
% ------------------------------------------------------------------

%\end{document}

\frontmatter
\maketitle{}

\begin{abstract}
When reading papers in areas that I am less familiar with, I often struggle to get from equation 12 to equation 13, despite the fact that \say{it follows trivially} for the authors. In such cases, it takes me some time and coffee to make the jump.  The purpose of this document is too keep track of some of these derivations so that when I re-read the paper, I suffer slightly less on the second pass\footnote{Note that it is \textbf{not} in any way a criticism of the authors of the papers, for whom the derivations were obvious.}. Any contributions, feedback or notifications of mistakes are much appreciated\footnote{Feel free to contact me by raising an issue on the github page, submitting a pull request directly (\mbox{\url{https://github.com/albanie/derivations})}}).
\end{abstract}
\clearpage

\tableofcontents*
\clearpage

\mainmatter
\chapter{Introduction}\label{sec:intro}

Calculus plays a central role in computer vision. As the community has integrated ever more closely with techniques from machine learning, statistics and optimisation, the ability to differentiate vector and matrix functions has become more useful to computer vision researchers.  Unfortunately, multivariate calculus, when performed with indices over element locations is a fairly tricky business to build an intuition for.  Although the index-based approach has the majority of the market share in the research world, there is an alternative: the slightly lesser known method of \textit{matrix differentials}.  We have found this alternative approach to be significantly simpler, more intuitive and faster to work with.  

Finally, we note that the widespread adoption of automatic differentiation (autodiff) has been a major boon to the whole community, in many cases obviating the need to perform arduous pen and paper derivations.  The goal of these notes is not to encourage the reader to avoid using autodiff, but to assist them in understanding, on a mathematical level, what is going on \say{under the hood}.


\section{Useful Resources}

There are a number of extremely good references for getting to grips with the calculus of vector and matrix functions.  Below are a few that we have found particularly useful, providing much of the material for these notes:

\begin{itemize}
\item Written by econometricians Magnus and Neudecker (originally in $1988$, but revised several times), \textit{Matrix differential calculus with applications in statistics and econometrics} is the canonical reference for matrix differential calculus~\cite{magnus1988matrix}.
\item The fundamentals of working with matrices \cite{searle2017matrix}.
\item \cite{kinghorn1996integrals}, \cite{minka2000old} and \cite{petersen2008matrix} each contain a large number of highly useful results and are good as quick references.
\item The MatConvNet manual \cite{vedaldi2015matconvnetmanual} provides carefully worked through examples of matrix derivatives.  While the emphasis is primarily on index-based derivations, it provides the derivatives of many common neural network computation blocks in matrix form and is very useful as a reference. 
\end{itemize}
\chapter{Machine Learning} \label{chap:ml}

\subsection{Understanding deep learning requires rethinking generalisation \cite{zhang2016understanding}, Theorem 1}

Link to paper: \url{https://arxiv.org/abs/1611.03530}. \\

This paper contains a nice theorem stating that a two-layer neural network containing a ReLU is capable of mapping $n$ samples $\{\mathbf{x}_i\}_{i=1}^n$ where $\mathbf{x}_i \in \real^d$ to $n$ arbitrary labels $\{y_i\}_{i=1}^n$ using only $2n + d$ parameters.  In the paper, it is stated like this:

\papersays{We say that a neural network $C$ can represent any function of a sample of size $n$ in $d$ dimensions if for every sample $S \subset \real^d$ with $|S| = n$ and every function $f: S \to \real $, there exists a setting of the weights of $C$ such that $C(x) = f(x)$ for every $x \in S$. \\
\textbf{Theorem 1.} There exists a two-layer neural network with ReLU activations and $2n + d$ weights that can represent any function on a sample of size $n$ in $d$ dimensions.
}

They also give an outline of a proof for this theorem in the appendix. It begins with this lemma:

\papersays{\textbf{Lemma 1.} For any two interleaving sequences of $n$ real numbers $b_1 < x_1 < b_2 < x_2 < \dots < b_n < x_n$, the $n \times n $ matrix $A = [\max \{x_i - b_j, 0 \}]_{ij}$ has full rank.  Its smallest eigenvalue is $\min_i x_i - b_i$.}

The proof (given in the paper) of this can be seen from writing out the matrix definition and using the fact that $x_i > b_i$:

\begin{align}
A = \max \Bigg(
\begin{bmatrix}
  x_1 - b_1 & x_1 - b_2 & \hdots & x_1 - b_n \\
  x_2 - b_1 & x_2 - b_2 & \hdots & x_2 - b_n \\
  \vdots & \vdots & \ddots & \vdots \\
  x_n - b_1 & x_n - b_2 & \hdots & x_n - b_n \\
\end{bmatrix}
, 0 \Bigg) = 
\begin{bmatrix}
  x_1 - b_1 & 0 & \hdots & 0 \\
  x_2 - b_1 & x_2 - b_2 & \hdots & 0 \\
  \vdots & \vdots & \ddots & \vdots \\
  x_n - b_1 & x_n - b_2 & \hdots & x_n - b_n \\
\end{bmatrix}
\end{align}

So $A$ is lower-triangular, and therefore full-rank (so invertible).  The eigenvalues of a lower-triangular matrix matrix lie on the diagonal, so the smallest eigenvalue is the smallest element on the diagonal, which is $\min_i (x_i - b_i)$.

Next the paper gives the following proof of Theorem 1.  For weights $\mathbf{w}, \mathbf{b} \in \real^n$ and $\mathbf{a} \in \real^d$, consider the function $c: \real^n \to \real$, 

\papersays{
\begin{align}
c(\mathbf{x}) = \sum_{j=1} w_j \max \{ \langle \mathbf{a}, \mathbf{x} \rangle - b_j , 0\} \label{eqn:target-fn}
\end{align}
}

\papersays{It is now easy to see that $c$ can be expressed by a depth $2$ network with ReLU activations.} \\

After the sentence above, there is an explanation in the original paper.  The questions I had here were: (1) How do we pick the values of $\ba$ to get the desired behaviour in matrix $A$ and (2) What does the shape of the network look like? \\

Let's first work out (1), i.e. how to construct the vector $\ba \in \real^d$. We need to project each $d$-dimensional sample $\xx_i$ to a scalar $x_i$ such that we get a sequence $x_1 < x_2 < \dots < x_n$. Once we've done this, we can just choose n  scalar values $\{b_i: i \in \{1, \dots, n\}\}$ by picking each $b_i \in (x_{i-1}, x_i)$ to get the interleaved sequences.  One way to do the projection is to first arrange the samples $\{ x_i \}_{i=1}^n$ so that they are ordered in increasing order lexicographically (i.e. order the element by the first dimension, then break any ties using the second dimension, then the third etc.).  Then take $k$ to be the maximum absolute scalar value that appears in any of the dimensions of any of the samples ($k = \max_s \max_j |x_s^j|$, where $j$ indexes the dimension and $s$ indexes the sample number). Then we can define $\ba = (k^0, \dots, k^{d-1})^T \in \real^d$, so that $x_i = \ba^\xx_i$ produces an increasing sequence.

For (2), define a vector $\mathbf{y} \in \real^n$ containing the target value for each sample $\mathbf{x}_i$ (i.e. $y_i = f(\mathbf{x}_i)$).  We want to build a function $c$ that maps each input sample $\mathbf{x}_i$ to the corresponding element $y_i$.  We know from \textbf{Lemma 1} that $A$ has full rank, so we know that there is a unique solution to the equation $A \ww = \mathbf{y}$, given by $\ww = A^{-1}\mathbf{y}$. We are now ready to define our network with two linear layers and a ReLU as follows:

% Based on Andrea's diagrams

\begin{figure}
\begin{center}
\begin{tikzpicture}[auto, node distance=2cm]
\node (x0)  [data] {$\xx_i^{(0)}\in\real^d$};
\node (f1) [block,right of=x0]{$f_1$};
\node (f2) [block,right of=f1,node distance=3.5cm]{ReLU};
%\node (dots) [right of=f2]{...};
\node (fL) [block,right of=f2,node distance=3.5cm]{$f_2$};
\node (w1) [data, below of=f1] {$\hat{\ba} \in \real^{d \times n}, \mathbf{b} \in \real^n$};
%\node (w2) [data, below of=f2] {$\ww_2$};
\node (wL) [data, below of=fL] {$\ww \in \real^n$};
\node (cX) [data, right of=fL] {$c(\xx)\in\real$};
\draw [to] (x0.east) -- (f1.west) {};
\draw [to] (f1.east) -- node {$\xx_i^{(1)} \in \real^n$} (f2.west);
\draw [to] (f2.east) -- node {$\xx_i^{(2)} \in \real^n$} (fL.west) {};
%\draw [to] (dots.east) -- node {$\xx_{L-1}$} (fL.west) {};{
\draw [to] (fL.east) -- (cX.west) {};
\draw [to] (w1.north) -- (f1.south) {};
%\draw [to] (w2.north) -- (f2.south) {};
\draw [to] (wL.north) -- (fL.south) {};
\end{tikzpicture}
\end{center}
\caption{The two layer network for memorising $n$ samples.}
\end{figure}
Here the superscript $\xx_i^{(j)}$ just means the output of the $j^{\text{th}}$ layer for input sample $\xx_i$. Here, $f_1(\xx) = \hat{\ba}^T \xx - \bb$ and $f_2(\xx) = \ww^T \xx$.  Although the shape of the matrix $\hat{\ba}$ is $d \times n$, the columns are simply repeats of the vector $\ba$ i.e. $\hat{\ba} = \ba \bone_n^T$ (where we have used $\bone_n \in \real^n$ to denote a column of ones), so that $f_1$ only requires $d + n$, rather than $nd + n$ parameters.  We can now write the function $c(\xx)$ defined in Eqn.~\ref{eqn:target-fn} as:

\begin{align}
c(\xx) = (f_2 \circ \text{ReLU} \circ f_1) (\xx) = \ww^T(\text{ReLU}(\hat{\ba}^T \xx - \mathbf{b}))
\end{align}

Next, we can walk through the proof of Corollary 1, which appears next in the paper:

\papersays{\textbf{Corollary 1.} For every $k \geq 2$, there exists a neural network with ReLU activations of depth $k$, width $\mathcal{O}(n/k)$ and $\mathcal{O}(n+d)$ weights that can represent any function on a sample of size $n$ in $d$ dimensions.}

The proof in the paper describes the approach: first, without loss of generality, assume that the outputs of the first layer lie in $[0,1]$ (i.e. using the notation for the two-layer network,  $\forall i x_i \in [0, 1]$.  The basic idea is to split the input space into $b$ intervals and apply the proof from Theorem 1 to each of the intervals.  Then all that remains is to multiplex the outputs of the subnetworks responsible for each interval to create the final architecture.  The resulting architecture is shown in Figure~\ref{fig:k-layers}.

\begin{figure}
\begin{center}
\begin{tikzpicture}[auto, node distance=2cm]
% nodes
\node (x0)  [data] {$\xx_i^{(0)}\in\real^d$};
\node (f1) [block,right of=x0]{$f_1$};
\node (relu1) [block,right of=f1,below of=f1, node distance=2cm]{ReLU};
\node (relu2) [block,right of=relu1, node distance=2cm]{ReLU};
\node (dots) [right of=relu2,node distance=2cm]{...};
\node (relub) [block,right of=dots, node distance=2cm]{ReLU};
\node (f2) [block,below of=relu1,node distance=2cm]{$f_2$};
\node (f3) [block,below of=relu2,node distance=2cm]{$f_3$};
\node (dots) [right of=f3,node distance=2cm]{...};
\node (fbp1) [block,below of=relub,node distance=2cm]{$f_{b+1}$};
\node (m1) [block,below of=f2,node distance=2cm]{$m_1$};
\node (relum1) [block,right of=m1,node distance=2cm]{ReLU};
\node (m2) [block,right of=relum1,node distance=2cm]{$m_2$};
\node (relum2) [block,right of=m2,node distance=2cm]{ReLU};
\node (m3) [block,right of=relum2,node distance=2cm]{$m_2$};
\node (cX) [data, right of=m3] {$c(\xx)\in\real$};
% arrows
\draw [to] (x0.east) -- (f1.west) {};
\draw [to] (f1.east) -- (relu1.north);
\draw [to] (f1.east) -- (relu2.north);
\draw [to] (f1.east) -- (relub.north);
\draw [to] (relu1.south) -- (f2.north);
\draw [to] (relu2.south) -- (f3.north);
\draw [to] (relub.south) -- (fbp1.north);
\draw [to] (f2.south) -- (m1.north);
\draw [to] (f3.south) -- (m1.north);
\draw [to] (fbp1.south) -- (m1.north);
\draw [to] (m1.east) -- (relum1.west) {};
\draw [to] (relum1.east) -- (m2.west) {};
\draw [to] (m2.east) -- (relum2.west) {};
\draw [to] (relum2.east) -- (m3.west) {};
\draw [to] (m3.east) -- (cX.west) {};
%
%\node (w1) [data, below of=f1] {$\hat{\ba} \in \real^{d \times n}, \mathbf{b} \in \real^n$};
%\node (w2) [data, below of=f2] {$\ww_2$};
%\node (wL) [data, below of=fL] {$\ww \in \real^n$};
%\node (xL) [data, right of=fL] {$c(\xx)\in\real$};
%
%\draw [to] (f2.south) -- node {$\xx_i^{(1)} \in \real^n$} (fbp1.north);
%\draw [to] (f2.east) -- node {$\xx_i^{(2)} \in \real^n$} (fL.west) {};
%\draw [to] (dots.east) -- node {$\xx_{L-1}$} (fL.west) {};{
%\draw [to] (fL.east) -- (xL.west) {};
%\draw [to] (w1.north) -- (f1.south) {};
%\draw [to] (w2.north) -- (f2.south) {};
%\draw [to] (wL.north) -- (fL.south) {};
\end{tikzpicture}
\end{center}
\caption{The multilayer layer construction. \label{fig:k-layers}}
\end{figure}

The output of each of the $b$ subnetworks is a scalar, which memorises the labels for the input samples that fall into its assigned interval.  We can use two ReLU layers to construct indicator functions that ensure that the output of the overall network routes input samples from the correct subnetwork.

Suppose that, given some interval $[d, e]$,  we want to create a function $g(x)$ such that $g(x) = x \forall x \in [d, e]$ and $g(x) = 0$ otherwise. We can use the following function:

\begin{align}
  g(x) = e - \text{ReLU}(e - (\text{ReLU}(x - d) - d))
\end{align}

By concatenating the outputs of $f_2, f_3, \dots f_{b+1}$ into a $b$ dimensional 


\section{Acknowledgements}

This document uses the template design from the MatConvNet manual by Andrea Vedaldi, Karel Lenc and Ankush Gupta.

% ------------------------------------------------------------------
\bibliographystyle{abbrv}
\bibliography{../bib/refs.bib}
\end{document}
% ------------------------------------------------------------------

