\chapter{Scale Space} \label{chap:scale-space}

One of the fundamental challenges in understanding how humans perceive the world is to figure out how they handle the problem of scale.  Given low-level light information, they are able to efficiently parse textures, objects and scenes occur across a wide range of scales.  This is a highly non-trivial ability.  

Consider the problem of understanding an image from a line sketch.  The lines which define important details for objects at one scale (such as the grains on a piece of wood) are irrelevant at another scale, such as a picture of a forrest.  How can we represent information at multiple scales in a way that captures semantically meaningful content in an efficient manner?

There have been several major contributions in this area, but perhaps the most important was the introduction of \textit{scale space filtering} \cite{witkin1987scale}.