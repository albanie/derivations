\chapter{Support Vector Machines} \label{sec:svms}

Support Vector Machines (SVMs) formed one of the dominant machine learning classification approaches in computer vision for a sustained period of time (late 90s to early 2010s onwards).  They are still widely used for a number of problems (although they often no longer represent the state-of-the-art in cases where large quantities of training data are available).  The overview here is based on the course notes by Andrew Ng \cite{svmsNg}.

The basic ideas of SVMs are easiest to understand in the context of binary classification (although they can be extended to more complex outputs with \textit{structural svms}).   We assume we are given a set of samples $\{\bx_i, y_i\}_{i=1}^n, \bx_i \in \real^d, y_i \in \{-1, 1\}$ drawn i.i.d. from the joint distribution $p(\bx,y)$.  We proceed to learn a model through risk minimisation \cite{vapnik1992principles}, which is to say that we aim to construct a model $f(\bx)$ that minimises the \textit{risk} of $f$ under this distribution:

\begin{align}
R(f) = \EE[l(f(\bx, \bw), y)] \label{eqn:risk}
\end{align}

where $\bw$ represent the parameters of the function. In practice, we cannot compute the expectation in Eqn.~\ref{eqn:risk} because we do not have access to the full underlying joint distribution $p(\bx, y)$.  However, since we have assumed that we have access to a handful of samples, we can approximate this quantity by the \textit{empirical risk}:

\begin{align}
R_{\text{emp}}(f) = \frac{1}{n}\sum_{i=1}^n l(f(\bx_i, \bw), y_i) \label{eqn:empirical-risk}
\end{align}

For the linear SVM, we formulate our predictive function $f(\bx, \bw)$ as the composition of a simple linear classifier and a thresholding function, $g$:

\begin{align}
f(\bx, \bw) = g(\bw^Tx + b) \label{eqn:linear-svm}
\end{align}

Here $b$ is a scalar bias term.  We could have included it directly into the parameter vector $\bw$ as an extra component and appended $1$ to the vector of inputs $\bx$, but for developing an intuition it can be helpful to separate it out.  Technically we should write $f(\bx, \bw, b)$ in the LHS of Eqn.~\ref{eqn:linear-svm}, but we omit the $b$ to keep the notation simpler). The threshold operator $g$ is defined as follows:

\begin{align}
g(x) =  \begin{cases}
  \mbox{ 1}, & x \geq 0 \\
 -1, & x < 0
 \end{cases} 
\end{align}

%% Keep as rough notes, but do not include in the final copy

In this derivation we assume the data points are linearly separable. Our aim is to pick the parameter values $(\bw, b)$ so as to minimise Eqn.~\ref{eqn:empirical-risk}.  In essence, that means that we should pick these values so that if $y_i = 1$, then our parameters should map $\bx_i$ to a positive value and to a negative value for pairs where $y_i = -1$. For a given choice of $(\bw, b)$ We define the functional margin of $\bx_i$ to be:

\begin{align}
\hat{\gamma}_i = y_i (\bw^T \bx_i + b) \in \real
\end{align}

Note that if the example is classified correctly, this margin will be positive. We define the function margin as:

\begin{align}
\hat{\gamma} = \min_i \hat{\gamma}_i
\end{align}

We would like to achieve the largest margin possible (this can be viewed as a proxy for better generalisation, where every point is classified correctly \say{safely}).  Next, we introduce a closely related quantity, the \textit{geometric margin} $\gamma_i$ for example pair $(\bx_i, y_i)$, which represents the distance from a training example to the separating hyperplane. We can compute this distance with a bit of simple geometry using the fact that

%\samsays{Add tikz diagram to show the intuition for this formula}

\begin{align}
\bw^T\Big(\bx_i - \gamma_i y_i \frac{\bw}{ ||\bw ||} \Big) = -b_i
\end{align}

which rearranges to give:

\begin{align}
\gamma_i = \frac{y_i}{|| \bw ||}\Big(\bw^T \bx_i + b_i\Big) = \frac{\hat{\gamma}_i}{||\bw||}
\end{align}

As above, we can also define the geometric margin of the function to be:

\begin{align}
\gamma = \min_i \gamma_i
\end{align}

We can now solve the following constrained optimisation to maximise the geometric margin:

\begin{align}
\max_{\bw, \gamma} \gamma & \\
\mbox{\text{s.t }  } &  y_i (\bw^T \bw_i + b) \geq \gamma, \quad i \in \{1, \dots, n\} \\
\mbox{\text{s.t }  } & ||\bw|| = 1
\end{align}

We are confronted with a slightly tricky issue.  We need the final constraint to prevent the algorithm just blowing up the parameters to solve the problem.  However, by forcing the weights to lie on the edge of a circle, the problem is no longer convex. We can resolve this by maximising the geometric margin, rather than the function margin:

\begin{align}
\max_{\bw, \hat{\gamma}} \frac{\hat{\gamma}}{||\bw||} & \\
\mbox{\text{s.t }  } &  y_i (\bw^T \bw_i + b) \geq \hat{\gamma}, \quad i \in \{1, \dots, n\}
\end{align}

We can apply an arbitrary scaling of the function margin without affecting the classifier, so we constrain the  \textit{function margin} to be $\hat{\gamma} = 1$.  Now, we have the problem

\begin{align}
\max_{\bw} \frac{1}{||\bw||} & \\
\mbox{\text{s.t }  } &  y_i (\bw^T \bw_i + b) \geq 1, \quad i \in \{1, \dots, n\}
\end{align}

Finally we can rewrite this to derive an equivalent maximisation. exclude

We can solve for the model parameters by solving the following quadratic program (i.e. a convex, quadratic objective and linear constraints):

\begin{align}
\min_\bw \frac{1}{2}||\bw||^2 & \\
\mbox{\text{s.t }  } &  y_i (\bw^T \bx_i + b) \geq 1, \quad i \in \{1, \dots, n\}
\end{align}

This is now a quadratic program (it has a convex, quadratic objective and linear constraints).

% more rough notes, should also be excluded
While this optimisation provides one way to solve for the model parameters, there is an alternative approach based on Lagrangian-duality which can offer better optimisation characteristics for certain problems.  Suppose we have a constrained optimisation problem of the form:

\begin{align}
\min_{\bw} f(\bw) & \\
\mbox{\text{s.t }  } &  \mathbf{g}(\bw) \preceq \mathbf{0}_k \in \real^k \\
\mbox{\text{s.t }  } & \bh(\bw) = \mathbf{0}_l \in \real^l
\end{align}

here we have encoded a set of $k$ inequalities as a single vector constrain ($\bx \preceq \by$ means that all elements of $\bx$ are less than or equal to all elements of $\by$) and $\mathbf{0}_m \in \real^m$ denotes a column vector of zeros.  We have similarly encoded a set of $l$ equalities as  a single vector equation.  We can solve this problem with the method of Lagrange multipliers, which introduces the generalised Lagrangian, together with additional parameters $\bu \in \real^k$ and $\bv \in \real^l$:

\begin{align}
\mathcal{L}(\bw, \bu, \bv) = f(\bw) + \bu^T \mathbf{g}(\bw) + \bv^T \mathbf{h}(\bw)
\end{align}

We can re-write the primal problem as:

\begin{align}
\theta_{\mathcal{P}}(\bw) = \max_{\bu, \bw, \bu \succeq \mathbf{0}_k} \mathcal{L}(\bw, \bu, \bv)
\end{align}

If any of the constraints are violated, then this maximisation will take the value $\infty$, otherwise it will take the value $f(\bw)$.  Therefore, our aim is to find the parameters that satisfy the following \textit{min-max} problem:

\begin{align}
\min_{\bw} \theta_{\mathcal{P}}(\bw) = \min_{\bw} \max_{\bu, \bw, \bu \succeq \mathbf{0}_k} \mathcal{L}(\bw, \bu, \bv)
\end{align}
 
if we write $p*$ for the solution of this problem, we note that we can also consider the symmetric \say{dual} problem, of the form:

\begin{align}
d^* = \max_{\bu, \bv} \theta_{\mathcal{D}}(\bu, \bw) = \max_{\bu, \bv, \bu \succeq \mathbf{0}_k} \min_{\bw} \mathcal{L}(\bw, \bu, \bv)
\end{align}

We know from the \textit{max-min} inequality that $d^* \leq p^*$, with equality only in special cases\footnote{One set of conditions required for equality is given by the first minimax theorem, proved by von Neumann in his seminal paper on game theory in 1928 \cite{neumann1928theorie}.}.  The conditions required for equality are that $f$ and $\mathbf{g}$ are convex, $\mathbf{g}$ is strictly feasible (there exists some $\bw$ for which $\mathbf{g}(\bx, \bw) \prec \mathbf{0}_k$) and $\bh$ is affine. In these cases, we can find a solution $\bw^*, \bu^*, \bv^*$ where $\bw^*$ solves the primal problem and $\bu^*, \bv^*$ solve the dual problem (and $p^* = d^* = \mathcal{L}(\bw^*, \bu^*, \bv^*)$).  Importantly, this solution will satisfy the \textit{Karush-Kuhn-Tucker} (KKT) conditions:

\begin{align}
\nabla_{\bw} \mathcal{L}(\bw^*, \bu^*, \bv^*) = \, & \mathbf{0} \in \real^d \\
\nabla_{\bv} \mathcal{L}(\bw^*, \bu^*, \bv^*) = \, & \mathbf{0} \in \real^l \\
\bu^* \odot \mathbf{g}(\bw^*) = \, & \mathbf{0} \in \real^k \quad \text{(dual complementarity condition)} \label{eqn:dual-complement}\\
\mathbf{g}(\bw^*) \preceq \, & \mathbf{0} \in \real^k \\
\bu^* \succeq \, & \mathbf{0} \in \real^k
\end{align}

The dual complementarity condition indicates that only a subset of the constraints need to be \say{active}, i.e. with $u_i > 0$ for some $i \in \{1, \dots, k\}$.

We can apply this approach directly to the SVMs problem formulation, where:

\begin{align}
f(\bw) = \frac{1}{2}||\bw||^2 \\
\mathbf{g}(\bw) = (\mathbf{1} - \by \odot (X^T\bw + b\mathbf{1}_n)) \preceq \mathbf{0}_n \in \real^{n}
\end{align}

where $\mathbf{1}_n, \mathbf{0}_n \in \real^n$ are column vectors of ones and zeros respectively and we have stacked the training samples into a matrix $X \in \real^{d \times n}$ with labels $\by \in \real^{n}$.  We form the generalised Lagrangian as before, noting that both the objective $f$ and the inequalities encoded in $g$ are convex functions:

\begin{align}
\mathcal{L}(\bw, \bu) = \frac{1}{2}||\bw||^2 + \bu^T(\mathbf{1}_n - \by \odot (X^T\bw + b\mathbf{1}_n)) \label{eqn:generalised-lagrangian}
\end{align}

so we know that we can solve for the solution of the primal and dual problems using the KKT conditions.  Using differentials, we can compute the gradient of $\mathcal{L}(\bw^*, \bu^*)$ with respect to $\bw$ as follows:

\begin{align}
{\dd}\mathcal{L} &= {\dd}\Big(\frac{1}{2}||\bw||^2\Big) + {\dd}\Big((\bu)^T(\mathbf{1}_n - \by \odot (X^T\bw + b\mathbf{1}_n))\Big) \\
&= \bw^T{\dd}\bw - \bu^T \underbrace{\by \odot X^T {\dd} \bw}_{n \times 1}  \\
&= \Big(\bw^T - \bu^T \by \odot X^T \Big) {\dd} \bw 
\end{align}

From which it follows that $\nabla_\bw \mathcal{L}(\bw, \bu) = \bw - X (\by \odot \bu) \in \real^d$.  Applying the first KKT condition (i.e. $\nabla_\bw \mathcal{L}(\bw, \bu) = \mathbf{0}_d$, we have that 

\begin{align}
\bw^{*} = X (\by \odot \bu) \label{eqn:dual-optimum}
\end{align}

As a result, the optimal solution for the weights lies in the span of the input samples. We can also take the gradient with respect to $b$ and set to zero to give that $\bu^T \by = 0$.  Plugging the first result back into Eqn.~\ref{eqn:generalised-lagrangian} gives:

\begin{align}
\mathcal{L}(\bw^*, \bu) = \frac{1}{2}  ||X \by \odot \bu ||^2  + \bu \odot  (\mathbf{1} - \by \odot (X^T X (\by \odot \bu) + b\mathbf{1}_n))
\end{align}

and we can use that $\bu^T \by = 0$ to cancel the last term:

\begin{align}
\mathcal{L}(\bw^*, \bu) &= \frac{1}{2} ||X \by \odot \bu ||^2 + \bu \odot  (\mathbf{1} - \by \odot X^T X (\by \odot \bu)) \\
&= \frac{1}{2} ||X \by \odot \bu ||^2 + \bu - (\by \odot \bu X^T)( X\by \odot \bu) \\
&= \frac{1}{2} ||X \by \odot \bu ||^2 + \bu - || X\by \odot \bu||^2 \\
&= \bu - \frac{1}{2} || X\by \odot \bu||^2 
\end{align}

We see that our derived conditions for solving the dual problem are:

\begin{align}
\max_{\bu} \mathcal{L}(\bw^*, \bu) &=   \bu - \frac{1}{2} || X\by \odot \bu||^2 \\
\text{s.t.} \quad & \bu^T \by = 0 \\
\text{and} \quad & \bu \succeq \mathbf{0}_n
\end{align}

This is again a convex objective with affine constraints, so we know that the solution of the dual is equivalent to the solution of the primal and therefore we can solve this set of equations instead of the primal problem.  Having found the optimal value of $\bw$, we can also find the optimal value of $b$ by noting that at each \textit{active constraint}, we will have that $g_i(\bw) = 0$ for some $i \in \{1, \dots, n\}$. In these cases, $y_i (\bx_i^T \bw + b) = 1$, so that $(\bx_i^T \bw + b) = 1$ when the target label is $y_i = 1$ and $(\bx_j^T \bw + b) = -1$ when $y_j = -1$. By summing these constraints, we get that $(\bx_j + \bx_i)^T \bw + 2b = 0$ or, to write things more precisely, $b = \frac{1}{2}(\min_{i, y_i = 1} \bx_i^T\bw + \max_{i, y_i = -1} \bx_i^T\bw)$.

\section*{Kernels}

To make predictions with our model $y = \bw^T\bx + b$, we can insert the learned value of $\bw$ (given Eqn.~\ref{eqn:dual-optimum}) to see that:

\begin{align}
y = \underbrace{(\by^T \odot \bu^T)}_{1 \times n} \underbrace{\vphantom{(} X^T}_{n \times d} \underbrace{\vphantom{(} \bx}_{d \times 1} + b
\end{align}

so that predictions are simply a weighted sum of inner products between training examples.  The cost of this inner product is $\mathcal{O}(d)$ for each element in the weighted sum. This formulation allows us to exploit the powerful relationship between inner products and positive semi-definite kernels. We know from functional analysis that each Reproducing Kernel Hilbert Space







